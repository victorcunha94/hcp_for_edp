\documentclass[]{report}
\usepackage[utf8]{inputenc}
\usepackage{amsmath}
\usepackage{amsfonts}
\usepackage{amssymb}
\usepackage{graphicx}
\usepackage{float}
\usepackage{geometry}
\usepackage{booktabs}
\usepackage{multirow}

% Title Page
\title{}
\author{}


\begin{document}
\maketitle


\section{Formulação do Problema}

Considera-se a equação de Poisson bidimensional com coeficiente variável:

\begin{equation}
	-\nabla \cdot (K \nabla p) = f \quad \text{em} \quad \Omega
\end{equation}

onde:
\begin{itemize}
	\item $p = p(x,y)$ é a variável primária (pressão, temperatura, etc.)
	\item $K = K(x,y)$ é o coeficiente de difusividade
	\item $f = f(x,y)$ é o termo fonte
	\item $\Omega \subset \mathbb{R}^2$ é o domínio de estudo
\end{itemize}

\section{Discretização por Volumes Finitos}

\subsection{Malha Computacional}

Define-se uma malha estruturada com células $\Omega_{ij}$:

\begin{equation}
	\Omega_{ij} = [x_{i-\frac{1}{2}}, x_{i+\frac{1}{2}}] \times [y_{j-\frac{1}{2}}, y_{j+\frac{1}{2}}]
\end{equation}

com centro da célula em $(x_i, y_j)$.

\subsection{Integração sobre o Volume de Controle}

Integrando a equação~(1) sobre a célula $\Omega_{ij}$:

\begin{equation}
	-\iint_{\Omega_{ij}} \nabla \cdot (K \nabla p) \, dA = \iint_{\Omega_{ij}} f \, dA
\end{equation}

Aplicando o teorema da divergência:

\begin{equation}
	-\oint_{\partial \Omega_{ij}} (K \nabla p) \cdot \mathbf{n} \, ds = \iint_{\Omega_{ij}} f \, dA
\end{equation}

\subsection{Fluxos nas Interfaces}

O balanço de fluxos nas quatro faces da célula é dado por:

\begin{align}
	&F_{i+\frac{1}{2},j}^x + F_{i-\frac{1}{2},j}^x + F_{i,j+\frac{1}{2}}^y + F_{i,j-\frac{1}{2}}^y = f_{i,j} \Delta x_i \Delta y_j
\end{align}

onde os fluxos são aproximados por:

\begin{align}
	F_{i+\frac{1}{2},j}^x &\approx -K_{i+\frac{1}{2},j} \frac{p_{i+1,j} - p_{i,j}}{\Delta x_{i+\frac{1}{2}}} \Delta y_j \\
	F_{i-\frac{1}{2},j}^x &\approx +K_{i-\frac{1}{2},j} \frac{p_{i,j} - p_{i-1,j}}{\Delta x_{i-\frac{1}{2}}} \Delta y_j \\
	F_{i,j+\frac{1}{2}}^y &\approx -K_{i,j+\frac{1}{2}} \frac{p_{i,j+1} - p_{i,j}}{\Delta y_{j+\frac{1}{2}}} \Delta x_i \\
	F_{i,j-\frac{1}{2}}^y &\approx +K_{i,j-\frac{1}{2}} \frac{p_{i,j} - p_{i,j-1}}{\Delta y_{j-\frac{1}{2}}} \Delta x_i
\end{align}

\subsection{Equação Discretizada Final}

Substituindo as aproximações dos fluxos, obtém-se:

\begin{align}
	&-K_{i+\frac{1}{2},j} \frac{p_{i+1,j} - p_{i,j}}{\Delta x_{i+\frac{1}{2}}} \Delta y_j
	+ K_{i-\frac{1}{2},j} \frac{p_{i,j} - p_{i-1,j}}{\Delta x_{i-\frac{1}{2}}} \Delta y_j \nonumber \\
	&-K_{i,j+\frac{1}{2}} \frac{p_{i,j+1} - p_{i,j}}{\Delta y_{j+\frac{1}{2}}} \Delta x_i
	+ K_{i,j-\frac{1}{2}} \frac{p_{i,j} - p_{i,j-1}}{\Delta y_{j-\frac{1}{2}}} \Delta x_i
	= f_{i,j} \Delta x_i \Delta y_j
\end{align}

\subsection{Forma Matricial}

A equação discretizada pode ser escrita na forma matricial:

\begin{equation}
	\mathbf{A} \mathbf{p} = \mathbf{f}
\end{equation}

onde:
\begin{itemize}
	\item $\mathbf{A}$ é a matriz de coeficientes (esparsa)
	\item $\mathbf{p}$ é o vetor das incógnitas $p_{i,j}$
	\item $\mathbf{f}$ é o vetor do termo fonte
\end{itemize}

\section{Esquema Numérico}

O esquema utiliza:
\begin{itemize}
	\item \textbf{Diferenças centradas} para os gradientes
	\item \textbf{Avaliação harmônica} para $K_{i+\frac{1}{2},j}$ em meios heterogêneos
	\item \textbf{Malha não-uniforme} permitindo $\Delta x_i \neq \text{constante}$
\end{itemize}

\section{Conclusão}

O método de volumes finitos apresentado conserva localmente os fluxos e é adequado para problemas com coeficientes descontínuos e geometrias complexas.

\begin{abstract}
\end{abstract}

\end{document}          
